\section[Introduction]{Introduction}
\subsection[Geospatial Data]{Geospatial Data}


\begin{frame}{What is Geospatial data}
%\Fontvi
\begin{beamerboxesrounded}{}
	\begin{itemize}
		\item Combination of Geographical and Spatial information 
		\item Management of Geospatial data- Geographical Information System (GIS)
		\item Two data types, Vector and Raster 
		\item Spatial Information denoted by coordinates(x,y)(Longitude, Latitude)
		\item Vector- Point, Line, Polygon
		\item Raster- Bounding box
\end{itemize}
\end{beamerboxesrounded}
\end{frame}


\begin{frame}{Applications}
	%\Fontvi
	\begin{beamerboxesrounded}{}
		\begin{itemize}
			\item Integral in day to day life
			
			{\centering
			\includegraphics[scale=0.23]{system.png} }
			
			\item Quick Challenge! - Name any two companies 
		\end{itemize}
	\end{beamerboxesrounded}
\end{frame}


\begin{frame}{Application in Environmental or Atmospheric Sciences}
	%\Fontvi
	\begin{beamerboxesrounded}{Some applications}
		\begin{itemize}
			\item Asses global environmental change
			\item Land use/Land cover change  
			\item Pollution emission estimation
			\item Predicting weather and air quality
		\end{itemize}
	\end{beamerboxesrounded}
\end{frame}

\subsection[Python programming language]{Python programming language}

\begin{frame}{What is Python}
	%\Fontvi
	\begin{beamerboxesrounded}{}
		"I have used a combination of Perl, Fortran, NCL, Matlab, R and others for routine research, but found out this general- purpose language, Python, can handle almost all in an efficient way from requesting data from remote online sites to statistics, and graphics."
		by a scientist
	\end{beamerboxesrounded}
\end{frame}

\begin{frame}{Python - Characteristics}
	%\Fontvi
	\begin{beamerboxesrounded}{}
		\begin{itemize}
			\item Easy to read and learn
			\item High-level language
			\item General-purpose language
			\item Interpreted language
			\item Extensive libraries
			\item Large userbase 
		\end{itemize}
	\end{beamerboxesrounded}
\end{frame}

\begin{frame}{Anaconda- Python distribution}
	%\Fontvi
	\begin{beamerboxesrounded}{}
		\begin{itemize}
			\item Open source package manager
			\item Virtual environment 
			\item Extensively helped in improve user base
			\item Greater coverage on Data science and machine learning 
		\end{itemize}
	\end{beamerboxesrounded}
\end{frame}

\begin{frame}{Literal programming}
	%\Fontvi
	\begin{beamerboxesrounded}{}
		\begin{itemize}
			\item By Donald Knuth, Stanford University
			\item Readabable text with Code section
			\item Well suitaed for Demonstration, research, and teaching 
			\item A major shifit in writing code denoted with thought porcess and background information related to the code 
			\item Jupyter notebooks are for Literal programming
		\end{itemize}
	\end{beamerboxesrounded}
\end{frame}

